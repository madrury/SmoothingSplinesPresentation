\section{The Problems With Polynomials}
%
\begin{frame}
  Oftentimes choosing an appropriate degree polynomial will result in a  good looking fit:
  \begin{figure}
    \includegraphics[scale=0.08]{cubic_fit_to_training}
  \end{figure}
  Nonetheless, the class of polynomial functions has some properties undesirable for a candidate for general purpose smoothing.
\end{frame}
%
\begin{frame}
  Most obvious is the discrete nature of the classes $\mathcal{C}_d$: if a cubic smooth is underfit, while a quartic smooth is overfit, there is no where in the middle to go!
\end{frame}
%
\begin{frame}
  Less apparent is the \textbf{non-locality} of polynomial smoothers:
  \begin{figure}
    \includegraphics[scale=0.12]{cubics_fit_to_training}
  \end{figure}
\end{frame}
%
\begin{frame}
  Here we have changed the training data only in a small interval, yet the fit to the data has degraded in far away places.
  \begin{figure}
    \includegraphics[scale=0.08]{cubics_fit_to_training}
  \end{figure}
  This may be reasonable behavior in some cases, but is undesirable for a general method.
\end{frame}
%
\begin{frame}
  \begin{columns}
    \column{.7\textwidth}
      \begin{figure}
        \includegraphics[scale=0.12]{cubic_endpoint_behaviour}
      \end{figure}
    \column{.3\textwidth}
      Finally, polynomial fits often show edge effects, as they extrapolate a high degree curve beyond the bounds of the data:
  \end{columns}
\end{frame}
%
\begin{frame}
  \begin{columns}
    \column{.7\textwidth}
      \begin{figure}
        \includegraphics[scale=0.12]{high_degree_endpoint_behaviour}
      \end{figure}
    \column{.3\textwidth}
      The effect can be extreme for high dimensional fits:
  \end{columns}
\end{frame}
%
\begin{frame}
  The remainder of this talk will motivate and outline a general purpose method that resolves all of these issues.
\end{frame}	

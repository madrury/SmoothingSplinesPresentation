\documentclass{beamer}
\mode<presentation>{}

\usepackage{graphicx}
\graphicspath{ {/Users/matthewdrury/Presentations/smoothing_splines/plots/} }

\usepackage{amsmath}

\DeclareMathOperator*{\argmin}{arg\,min}
\DeclareMathOperator*{\bias}{bias}

\title{Smoothing Splines}
\author{Matthew Drury}

\begin{document}
%
\begin{frame}
  \titlepage
\end{frame}
%
\begin{frame}
  \frametitle{The Setup}
  \begin{figure}
    \includegraphics[scale=0.08]{training_data}
  \end{figure}
  Suppose we have some one dimensional data:
  $$\mathcal(D) = \left\{ (x_i, y_i) | i = 1,2,\ldots,N \right\}$$
\end{frame}
%
\begin{frame}
  Often our goal is to approximate a functional relationship that the data generating mechanism may obey:
  $$E(Y|X) \approx f^*(X)$$
  We will consistently use $f^*$ to notate an estimated function, and $f$ the ground truth function being estimated (unknowable in all real cases).
  
\end{frame}
%
\begin{frame}
  One way to operate is to choose some general collection of functions, and then attempt to determine the function in this collection that best approximates the data:
  $$ f^* = \argmin_{g} \left\{ \sum_i (x_i - g(x_i))^2 | g \in \mathcal{C} \right\} $$
  Choosing $\mathcal{C}$ to be progressively more encompassing will result in closer fits to the data.
\end{frame}
%
\begin{frame}
  For example, choosing $\mathcal{C}$ to be the collection of all linear functions:
  $$ \mathcal{C} = \left\{ x \mapsto ax + b | a,b \in \mathbb{R} \right\} $$
recovers classic linear regression.
  \begin{figure}
    \includegraphics[scale=0.08]{fitted_line}
  \end{figure}
\end{frame}
%
\begin{frame}
  \begin{figure}
    \includegraphics[scale=0.08]{fitted_line}
  \end{figure}
  There is a clear issue with the linear regression fit to our sample data set: there are places where \textbf{all} the data is far away from the fitted line.
\end{frame}
%
\begin{frame}
  A measurement of this defect is the \textbf{squared bias}:
  $$ \bias(f) = E_X \left[ E_{\mathcal{D}}[f^*(X)] - f(X) \right] ^2$$
  We say that a model suffering from a case of large bias is \textbf{under fit}.
\end{frame}
%
\end{document}
